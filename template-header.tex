\documentclass[12pt,a4paper]{article}

\usepackage{geometry}
\geometry{%
  a4paper,
  lmargin=1.7cm,
  rmargin=1.7cm,
  tmargin=2cm,
  bmargin=2cm,
  footskip=12pt,
  headheight=12pt}

\usepackage{amsmath}
\usepackage{amssymb}
\let\ge=\geqslant  \let\le=\leqslant
\usepackage{tikz}[2013/12/13 v3.0.0]
\usetikzlibrary{calc}
\tikzset{above/.default=6pt}
\tikzset{regular/.style={sloped, above}}
\tikzset{hidden/.style={sloped, above, fill=yellow!80!white}}

% Usage of these macros:
% First do: \setshapesize{size} where size is the side length of
% the triangle/square given in cm with no units
% Then:
% \...triangle{(x0,y0)}{label1}{label2}{label3}
% where the first argument is the coordinate of the centre,
% and the labels are placed anticlockwise in the following order:
% \uptriangle: base, right, left
% \downtriangle: base, left, right
% \lefttriangle: right, top, bottom
% \righttriangle: left, bottom, top

\newlength\trad % triangle circumradius
\newlength\srad % square circumradius
\newcommand{\setshapesize}[1]{%
 \pgfmathsetlength\trad{(#1*72.27/2.54) / sqrt(3)}
 \pgfmathsetlength\srad{(#1*72.27/2.54) / sqrt(2)}
}

\newcommand{\uptriangle}[6]{%
  \coordinate (a) at ($ #1 +(-30:\trad) $);
  \coordinate (b) at ($ #1 +(90:\trad) $);
  \coordinate (c) at ($ #1 +(-150:\trad) $);

  \draw[allow upside down]
    (a) -- node[sloped, above] {#3}
    (b) -- node[sloped, above] {#4}
    (c) -- node[sloped, above] {#2}
    cycle;
  \node[circle,draw,thin,rotate=#6] at ($ #1 $) {#5};
}
\newcommand{\downtriangle}[6]{%
  \coordinate (a) at ($ #1 +(30:\trad) $);
  \coordinate (b) at ($ #1 +(150:\trad) $);
  \coordinate (c) at ($ #1 +(-90:\trad) $);

  \draw[allow upside down]
    (a) -- node[sloped, above] {#2}
    (b) -- node[sloped, above] {#3}
    (c) -- node[sloped, above] {#4}
    cycle;
  \node[circle,draw,thin,rotate=#6] at ($ #1 $) {#5};
}
\newcommand{\lefttriangle}[6]{%
  \coordinate (a) at ($ #1 +(60:\trad) $);
  \coordinate (b) at ($ #1 +(180:\trad) $);
  \coordinate (c) at ($ #1 +(-60:\trad) $);

  \draw[allow upside down]
    (a) -- node[sloped, above] {#3}
    (b) -- node[sloped, above] {#4}
    (c) -- node[sloped, above] {#2}
    cycle;
  \node[circle,draw,thin,rotate=#6] at ($ #1 $) {#5};
}
\newcommand{\righttriangle}[6]{%
  \coordinate (a) at ($ #1 +(120:\trad) $);
  \coordinate (b) at ($ #1 +(-120:\trad) $);
  \coordinate (c) at ($ #1 +(0:\trad) $);

  \draw[allow upside down]
    (a) -- node[sloped, above] {#2}
    (b) -- node[sloped, above] {#3}
    (c) -- node[sloped, above] {#4}
    cycle;
  \node[circle,draw,thin,rotate=#6] at ($ #1 $) {#5};
}

\newcommand{\makestyles}[4]{%
  \def#1{#3}
  \def#2{#4}
}

\newcommand{\tiltedsquare}[8]{%
  \coordinate (a) at ($ #1 +({#2-135}:\srad) $);
  \coordinate (b) at ($ #1 +({#2-45}:\srad) $);
  \coordinate (c) at ($ #1 +({#2+45}:\srad) $);
  \coordinate (d) at ($ #1 +({#2+135}:\srad) $);

  \makestyles\stylea\contenta#3
  \makestyles\styleb\contentb#4
  \makestyles\stylec\contentc#5
  \makestyles\styled\contentd#6

  \draw[allow upside down]
    (a) -- node[\stylea] {\contenta}
    (b) -- node[\styleb] {\contentb}
    (c) -- node[\stylec] {\contentc}
    (d) -- node[\styled] {\contentd}
    cycle;
  \node[circle,draw,thin,rotate=#8] at ($ #1 $) {#7};
}

\newcommand{\upsquare}[7]{%
  \tiltedsquare{#1}{0}{#2}{#3}{#4}{#5}{#6}{#7}
}

\pagestyle{empty}


%%% Local Variables: 
%%% mode: latex
%%% TeX-master: t
%%% End: 
